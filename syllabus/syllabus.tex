% Física Computacional Syllabus
\documentclass[12pt]{article}
\usepackage[intlimits]{amsmath}
\usepackage{amsfonts}
\usepackage{amscd}
\usepackage{amssymb}
\usepackage{natbib}
\usepackage[spanish] {babel}

%\usepackage[latin1]{inputenc}
%\usepackage[utf8]{inputenc}


\title{FISI 2025 F\'isica Computacional \\
Semestre 2013 - 01\\
Martes y Jueves 1:00 - 2:20 \\
Salon Z 122 - Lab. CompuFis
}
\author{Profesor: Jaime Forero}

\begin{document}
\maketitle
\section*{Objetivo}


\section*{Software}
\noindent Se usar\'an principalmente los notebooks de IPython. Tambi\'en se aceptan tareas en FORTRAN 90/95, C, C++ y Python.

\section*{Evaluaci\'on}
Habr\'an 10 talleres para entregar, cada uno con un valor del 8\%. Habr\'a 2 quizes sorpresa, cada uno con un valor del 10\%. No habr\'a parciales ni examen final.

 
\section*{Programa}

\begin{center}
\begin{tabular}{|l|l|l|}
\hline
Semana 1 & &Consola/Emacs\\
Semana 2 & &C / Python *\\
Semana 3 & &Python / IPython Notebook *\\
Semana 4 & Matrices y sistemas de ecuaciones lineales & Data *\\
Semana 5 & M\'inimos cuadrados & Data *\\
Semana 6 & Interpolaci\'on & Version Control (Git) *\\
Semana 7 & Integraci\'on y derivaci\'on num\'erica & Version Control (Github) *\\
Semana 8 & An\'alisis de Fourier - Transformada de Fourier & Make *\\
Semana 9 & Soluci\'on de ecuaciones no lineales & Make\\
Semana 10 & Soluci\'on de ecuaciones no lineales & Testing\\
Semana 11 & Resoluci\'on de problemas con valor de frontera & Testing *\\
          & para ecuaciones diferenciales ordinarias & \\
Semana 12 & Ecuaciones diferenciales parciales & Dise\~no de programas\\
Semana 13 & M\'etodos Monte Carlo & Dise\~no de programas *\\
Semana 14 & M\'etodo de inferencia bayesiana& Bases de datos *\\
Semana 15 & M\'etodo de diferencias finitas & Bases de datos\\
\hline
\end{tabular}
\end{center}


\section*{Bibliograf\'ia}


\section*{P\'agina Web}
\begin{verbatim}
http://wwwprof.uniandes.edu.co/~je.forero/
https://github.com/forero/ComputationalPhysicsUniandes
\end{verbatim}

 

\end{document}
