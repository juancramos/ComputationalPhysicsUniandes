% Física Computacional Syllabus
\documentclass[12pt]{article}
\usepackage[intlimits]{amsmath}
\usepackage{amsfonts}
\usepackage{amscd}
\usepackage{amssymb}
\usepackage{natbib}
\usepackage[spanish] {babel}

%\usepackage[latin1]{inputenc}
%\usepackage[utf8]{inputenc}



\begin{document}
\noindent
FISI 2025 F\'isica Computacional \\
Semestre 2013 - 01\\
Martes y Jueves 1:00 - 2:20 \\
Salon Z 122 - Lab. CompuFis\\
Profesor: Jaime Forero, email: je.forero\\


\section*{Objetivo}
El curso tiene como objetivo principal \emph{desarrollar en los estudiantes una adecuada actitud computacional, con la capacidad de discernir sobre los m\'etodos adecuados para solucionar cualquier problema y entender sus limitaciones.}

En esta clase dar\'e enfasis a esa \emph{actitud computacional} que corresponde al conjunto de habilidades para trabajar con computadores en generar y procesar datos que correspondan a sistemas f\'isicos, donde estos datos corresponden a una medici\'on o una simulaci\'on.

\section*{Metodolog\'ia}
Esa \emph{actitud computacional} se desarrolla trabajando. Las sesiones de f\'isica computacional ser\'an, sobre todo, una sesi\'on de exploraci\'on, pr\'actica y experimentaci\'on. Para que esto funcione es necesario que los estudiantes lleguen a clase despu\'es de haber le\'ido sobre el tema correspondiente. El mismo d\'ia de clase se repartir\'an lecturas (por SicuaPlus) para la clase siguiente.

El programa del curso tiene dos partes bien diferenciadas. La parte de m\'etodos tradicionales de computo num\'erico y la parte de \emph{carpinter\'ia} de software. La primera es probable que le sea \'util a una fracci\'on de los asistentes al curso en su vida profesional. La segunda parte le ser\'a \'util a \emph{todos}.

Las lecturas que se deben completar antes de clase tienen que ver con la primera parte, sobre todo. Mientras que la segunda parte de carpinter\'ia ser\'a trabajada en clase mientras se resuelven problemas pr\'acticos de c\'omputo num\'erico.

\section*{Software}
\noindent Se usar\'an principalmente los notebooks de IPython complementado con C. Tambi\'en se aceptan tareas en los siguientes lenguajes de programaci\'on: FORTRAN 90/95, C++ y Python. No se aceptar\'an tareas en Matlab, Matem\'atica o cualquier otro lenguaje de programaci\'on que no este en la lista mencionada antes. 

\section*{Evaluaci\'on}
Hay 10 talleres para entregar, cada uno con un valor del 9\%. Habr\'a  quizes sorpresa, que contar\'an por el 10\% restante. No habr\'a parciales ni examen final.

 
\section*{Programa}

\begin{center}
\begin{tabular}{|l|l|l|c|}
\hline
Sem. & Teor\'ia & Carpinter\'ia & Taller \\\hline
1 & 	&Consola/Emacs &\\
2 & 	&C  & \#1\\
3 & 	&Python / IPython Notebook & \#2\\
4 & Matrices y sistemas de ecuaciones lineales  & Version Control (Github) &\#3\\
5 & M\'inimos cuadrados & Version Control (Github)& \#4\\
6 & Interpolaci\'on & Data & \\
7 & Integraci\'on y derivaci\'on num\'erica & Data &\\
8 & An\'alisis de Fourier - FFT  (FFTW)& Make& \#5 \\
9 & Ecuaciones diferenciales ordinarias & Make&\\
 & {\bf Semana de trabajo individual} & &\\
10 & Ecuaciones diferenciales ordinarias & Testing& \\
11 & Ecuaciones diferenciales ordinarias & Testing & \#6\\
12 & Ecuaciones diferenciales parciales & Dise\~no de programas&\#7\\
13 & M\'etodos Monte Carlo & Dise\~no de programas &\#8\\
14 & M\'etodo de diferencias finitas & C+Python &\#9\\
15 & M\'etodo de inferencia bayesiana& C+Python &\#10\\
\hline
\end{tabular}
\end{center}


\section*{Bibliograf\'ia}
\begin{itemize}
\item
A survey of Computational Physics. R. H. Landau, M. J. P\'aez, C. C.
Bordeianu. Princeton Univ. Press. 2008 
\item
Statistical Mechanics: Algorithms and Computations. W. Krauth, Oxford Univ. Press. 
\item\verb"http://software-carpentry.org/"
\end{itemize}

\section*{P\'agina Web}
\begin{verbatim}
http://forero.github.com/ComputationalPhysicsUniandes/
\end{verbatim}

 

\end{document}
