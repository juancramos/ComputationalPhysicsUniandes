\documentclass{article}
\usepackage{verbatim}
\usepackage{graphicx}

\begin{document}
\setcounter{section}{2}
\begin{center}
\includegraphics[scale=0.60]{python.png}
Source: {\texttt http://xkcd.com/353/}
\end{center}
\section{Basics of Python}
Python is a {\bf high-level} language. There are big advantages in
working with such languages. The most important is that source code is
faster to write and easier to read. You won't have to declare the type
of the variables or allocate the memory by yourself!

In these kind of languages the compiling process is absent. You start
writing source code and then feed-it into an interpreter which is in
charge of executing the program. The interpretation and execution
occur alternately. Remember that in the case of a {\bf low-level} language the
compiler reads and translates the source code before execution. 

Python source code is executed by an interpreter that you can call
from the terminal: 
\begin{verbatim}
$python
Python 2.7.3 (default, Aug  1 2012, 05:14:39) 
[GCC 4.6.3] on linux2
Type "help", "copyright", "credits" or "license" for more information. 
>>> 
\end{verbatim}

The prompt \verb">>>" indicates that the Python interpreter is ready
to receive instructions. After each instruction the interpreter
displays the result. For instance 

\begin{verbatim}
>>> 1
1
\end{verbatim}

or in the case of arithmetic operations

\begin{verbatim}
>>> 2+2
4
\end{verbatim}

You can also interact with the interpreter by writing all the lines
that you want to be interpreted into a text file with the extension
\verb".py". In that case the interpretation and execution can be called
from the terminal as follows. 

\begin{verbatim}
$ python mycode.py
\end{verbatim}

In this class we are interested in covering the very basics of python. The
recommended reference is this on-line text: 

\begin{itemize}
\item Think Python: How to Think Like a Computer Scientist, Allen
  B. Downey.\\\verb"http://www.greenteapress.com/thinkpython/html/index.html" 
\end{itemize}

After knowing the basics a very useful repository for useful recipes
in Python is here:

\begin{itemize}
\item Python Grimoire: \\\verb"https://taoofmac.com/media/dev/Python/Grimoire/grimoire.html"
\end{itemize}

In what follows I will go through basic concepts in python illustrated
by source code.

\subsubsection{My first Python program}

Save the following code in a file \verb"hello.py"
\verbatiminput{../hands_on/python/hello.py}

Then execute it as:
\begin{verbatim}
$python hello.py
\end{verbatim}

Compare it to the C source code. Smile.

\subsection{Variables and Statements}
You don't need to declare the type of the variable before using them. 
 
\verbatiminput{../hands_on/python/variables.py}


You can also perform simple arithmetic operations
\verbatiminput{../hands_on/python/arithmetic.py}

Advanced functions can should be used in the following way
\verbatiminput{../hands_on/python/math.py}


\subsection{Conditionals}
The most common way to control the flow with the program is with the
\verb"if", \verb"else", \verb"elif" and \verb"while" statements

\verbatiminput{../hands_on/python/if_while.py}


\subsection{Iteration}
Iteration comes in a flavor of \verb"for" loops:
\verbatiminput{../hands_on/python/print_table.py}


\subsection{Lists}
One of the most basic and useful structures in python are lists. They
can include any kind of variable.

A list can be defined as:
\begin{verbatim}
my_list = ["Apple", 3.40, 1, "Hello world"]
\end{verbatim}

the following code would print all the items in the previous list
\begin{verbatim}
print my_list[0]
print my_list[1]
print my_list[2]
print my_list[3]
\end{verbatim}

If you want to print the first item to the third

\begin{verbatim}
print my_list[0:3]
\end{verbatim}

Or if you want to iterate over each item in the list and do something
with them:

\begin{verbatim}
for item in list:
    new_item = 2 * item
    print new_item
\end{verbatim}

\subsection{Reading files}

\verbatiminput{../hands_on/python/read_file.py}

\subsection{Functions}
The definition of functions follows the generic form:

\begin{verbatim}
def function_name(inputs):
    statements
\end{verbatim}

The indentations of the statements is {\bf very important}.  This is
the way the interpreter recognizes the end and the beginning of your
function. Here is an example of two functions   


\verbatiminput{../hands_on/python/functions.py}

\end{document}
