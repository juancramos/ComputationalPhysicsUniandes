\documentclass{article}
\title{Taller \#2. F\'isica Computacional / FISI 2025 \\Semestre 2013-I. \\ Profesor: Jaime E. Forero Romero}
\date{Febrero 7 2013}
\begin{document}
\maketitle

{\bf Esta tarea debe estar en un repositorio de la cuenta de github de
  cada uno con un commit final hecho antes del medio d\'ia del jueves
  14 de Febrero del 2013} 

\begin{enumerate}
\item
Visiten la p\'agina \verb"http://www.gutenberg.org/catalog/". Vamos a
elegir algunos libros para trabajar. 

\item
Bajen la version de texto simple UTF-8 (no HTML, EPUB o Kindle) de un
libro en cada uno de los cinco idiomas siguientes:  
\begin{itemize}
\item ingl\'es: \verb"http://www.gutenberg.org/browse/languages/en"
\item espa\~nol: \verb"http://www.gutenberg.org/browse/languages/es"
\item finland\'es: \verb"http://www.gutenberg.org/browse/languages/fi"
\item italiano: \verb"http://www.gutenberg.org/browse/languages/it"
\item ingl\'es antiguo: \verb"http://www.gutenberg.org/browse/languages/ang"
\end{itemize}

\item
Escriba un programa en Python que cuente cu\'antos caracteres
diferentes hay en cada uno de los libros. Los caracteres con
di\'eresis o tildes cuentan como caracteres diferentes. Los signos de
puntuaci\'on no se deben incluir en la cuenta. 

\item 
Por cada caracter diferente calcule su frecuencia en el texto. Es decir, el
porcentaje de veces que aparece cada caracter con respecto
al n\'umero total de caracteres en el texto. 

\item 
En el caso de un libro llamado \verb"DonQuijoteDeLaMancha.txt"
el c\'odigo se debe poder ejecutar de la siguiente manera: 
\begin{verbatim}
python frecuencias.py DonQuijoteDeLaMancha.txt
\end{verbatim}

\item
El output del c\'odigo en Python debe ser un archivo de texto por cada
libro elegido. En el caso de este ejemplo, el archivo de salida se
debe llamar \verb"frecuencias_DonQuijoteDeLaMancha.txt".  

\item
En el archivo de texto se deben escribir dos columnas: en la primera
los caracteres del libro y en la segunda su frecuencia. Las filas
deben estar ordenadas en orden decreciente de frecuencia. 

\item
El codigo fuente de Python y los archivos de los libros deben estar en
un repositorio de GitHub. Dentro del repositorio, deben estar adentro
de un directorio que se llame 
\verb"python". 

\item
Si tienen algun comentario que hacer sobre las diferentes distribuciones
de caracteres en los idiomas estudiados lo pueden incluir en un archivo de
texto que se llame \verb"comentario.txt"

\item
Enviar un email la direcci\'on {\tt j.e.forero.romero} en {\tt
  gmail} con el subject
\verb"RESPUESTA TALLER 2 FISICA COMPUTACIONAL". En el cuerpo del texto
debe ir la direcci\'on del repositorio donde est\'a la tarea. 

\item
La calificaci\'on se har\'a en tres partes.
\begin{itemize}
\item El c\'odigo fuente y los libros est\'an en un repositorio de
  github (20\%). 
\item El c\'odigo funciona de manera adecuada con los cinco libros
  elegidos (60\%). 
\item El c\'odigo funciona de manera adecuada con un libro en
  esperanto de mi elecci\'on. (20\%) 
\end{itemize}
\end{enumerate}
\end{document}
